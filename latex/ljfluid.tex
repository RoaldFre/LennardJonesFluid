\documentclass[11pt,a4paper]{article}
\usepackage[english]{babel}
\usepackage{a4wide}
\usepackage[pdftex]{graphicx}
\usepackage{color}
\usepackage{placeins}
\usepackage{epstopdf}
\usepackage{amsfonts, amsmath, amsthm}
\usepackage{tabularx}
\usepackage[font=small,format=plain,labelfont=bf,up,textfont=up]{caption}

%Loosen up on figure placement restrictions
\renewcommand{\textfraction}{0.1}
\renewcommand{\topfraction}{0.9}
\renewcommand{\bottomfraction}{0.9}
\renewcommand{\floatpagefraction}{0.8} 

\begin{document}
\author{Roald Frederickx}
\title{Lennard-Jones fluid}
\date{}
\maketitle

\section{Explanation of parameters}
\begin{center}
%\begin{tabular}{c|p}
\begin{tabularx}{\linewidth}{c | X}
$N$&	The number of particles.\\[5pt]
$\rho$&	The volume density of the particles.\\[5pt]
$\Delta t$&	The time step.\\[5pt]
$T_0$&		The initial temperature. Initial velocity components are sampled from a gaussian distribution with variance $T_0$.\\[5pt]
$T$&		The temperature of the Berendsen thermostat.\\[5pt]
$\tau$&		The coupling time scale of the Berendsen thermostat. At each timestep, velocities will be rescaled by a factor $\lambda = \sqrt{1 + \Delta t/\tau\left(T_0/T_{\mathrm{kin}} - 1\right)}.$\\[5pt]
$\tau_{\mathrm{sample}}$&	Time between subsequent measurement samples of the system.\\[5pt]
$N_{\mathrm{samples}}$&		Total number of samples of the system (ie, the system is sampled for a total time ($\tau_\mathrm{sample} \cdot N_\mathrm{samples}$).\\
\end{tabularx}
\end{center}

\section{Energy plots}
Plots of the energy are shown for (1) an isolated system (canonical ensemble, no thermal coupling), (2) the same system connected to a thermal bath at fixed $T$ where the initial system has velocities sampled for an initial temperature $T_0 = T$ and (3) the system in contact with a thermal bath at temperature $T$, but the initial system starts with a very low temperature $T_0 \ll T$.

Each of these three plots is shown for varying density.

\input{images/energies-uncoupled-0p2}
\input{images/energies-0p2}
\input{images/energies-T0-0p2}

\input{images/energies-uncoupled-0p4}
\input{images/energies-0p4}
\input{images/energies-T0-0p4}

\input{images/energies-uncoupled-0p6}
\input{images/energies-0p6}
\input{images/energies-T0-0p6}

\input{images/energies-uncoupled-0p8}
\input{images/energies-0p8}
\input{images/energies-T0-0p8}


\FloatBarrier
\section{Pair correlation plot}
\input{images/pairCorr}


\FloatBarrier
\section{Pressure plot}
\input{images/pressure}

\end{document}
